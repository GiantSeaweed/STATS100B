%!TEX program = xelatex
\documentclass[a4papers]{ctexart}
%数学符号
\usepackage{amssymb}
\usepackage{amsmath}
%表格
\usepackage{graphicx,floatrow}
\usepackage{array}
\usepackage{booktabs}
\usepackage{makecell}
%页边距
\usepackage{geometry}
\geometry{left=2cm,right=2cm,top=2cm,bottom=2cm}

%首行缩进两字符 利用\indent \noindent进行控制
\usepackage{indentfirst}
\setlength{\parindent}{2em}

\setromanfont{Songti SC}
%\setromanfont{Heiti SC}

\title{STATS100B--Introduction to Mathematical Statistics \\Homework 7}
\author{Feng Shiwei \ UID:305256428}
\date{}
\begin{document}
\maketitle
\section*{Question a}
\noindent Solution:\\
\indent 
\[L=\dfrac{1}{(2\pi)^{\frac{n}{2}}}|\boldsymbol{\Sigma}|^{-\frac{1}{2}} 
    e^{\frac{1}{2}(\boldsymbol{\boldsymbol{\boldsymbol{Y}}}-\mu\boldsymbol{1})'\boldsymbol{\Sigma}^{-1}(\boldsymbol{\boldsymbol{Y}}-\mu\boldsymbol{1}) }
\]
\[lnL = -\dfrac{n}{2}ln(2\pi\boldsymbol{\Sigma}^2)-\dfrac{1}{2}ln|\boldsymbol{V}|-\dfrac{1}{2\sigma^2}(\boldsymbol{Y}-\mu)V^{-1}(\boldsymbol{Y}-\mu)
 \]
 \[\dfrac {\partial \ln L}{\partial \mu }=-\dfrac {1}{2\sigma ^{2}}\left[ -\boldsymbol{Y'V^{-1}1}-\boldsymbol{1'V^{-1}Y}+2\boldsymbol{\mu 1'V^{-1}1}\right] =0
     \]
\[ \therefore \hat{\mu} = \boldsymbol{ \dfrac{1'V^{-1}Y}{1'V^{-1}1} }\]
\[ \dfrac {\partial \ln L}{\partial \sigma ^{2}}=-\dfrac {n}{2\sigma ^{2}}+\dfrac {1}{\sigma ^{4}}\left( \boldsymbol{Y}-\mu \boldsymbol{1}\right)'\boldsymbol{V}^{-1}\left( \boldsymbol{Y}-\mu \boldsymbol{1}\right) =0\]
\[\therefore \hat{\sigma^2} =\dfrac{\left( \boldsymbol{Y}-\hat{\mu} \boldsymbol{1}\right) '\boldsymbol{V}^{-1}\left( \boldsymbol{Y}-\hat{\mu} \boldsymbol{1}\right)}{n} \]


\section*{Question b}
\noindent  Solution:
\[E(\hat{\mu}) = E\left(\dfrac{1'V^{-1}\boldsymbol{Y}}{1'V^{-1}1}\right) = \dfrac{1'V^{-1}E(\boldsymbol{Y})}{1'V^{-1}1} =\dfrac{1'V^{-1}\mu1}{1'V^{-1}1} =\mu\]
\[  \]



\section*{Question c}
\noindent Solution:
\[ \boldsymbol{I}(\boldsymbol{\theta}) = -E \begin{pmatrix}
    \dfrac {\partial ^{2}\ln L}{\partial \mu^2} 
    & \dfrac {\partial^{2}\ln L}{\partial \mu \partial \sigma ^{2}} \\ \\
    \dfrac {\partial ^{2}\ln L}{\partial \sigma ^{2}\partial \mu } 
    & \dfrac {\partial \ln L}{\partial {\sigma ^{2}}^{(2)} }
\end{pmatrix}\]
\[  \dfrac {\partial ^{2}\ln L}{\partial \mu^2 }=-\dfrac {1'V^{-1}1}{\sigma ^{2}} \]
\[ \dfrac {\partial^{2}\ln L}{\partial \mu \partial \sigma ^{2}}=\dfrac {1}{\sigma ^{4}}\left[ \mu 1'V^{-1}1-1'V^{-1}\boldsymbol{Y}\right] \]
\[ \dfrac {\partial ^{2}\ln L}{\partial \sigma ^{2}\partial \mu } = \dfrac {1}{\sigma ^{4}}\left[ \mu 1'V^{-1}1-1'V^{-1}\boldsymbol{Y}\right] \]
\[ \dfrac {\partial \ln L}{\partial {\sigma ^{2}}^{(2)} } =\dfrac {n}{2\sigma ^{4}}-\dfrac {1}{\sigma ^{6}}\left( Y-\mu 1\right) 'V^{-1}\left( \boldsymbol{Y}-\mu 1\right)  \]
\[ \therefore \boldsymbol{I}(\boldsymbol{\theta}) =\begin{pmatrix}

\end{pmatrix}  \]

\section*{Question d}
\noindent Solution:
\[ f\left( y_{i}\right) =\dfrac {1}{\sqrt {2\pi }i\sigma }e^{-\dfrac {\left( y-i\theta \right) ^{2}}{2\left( i\sigma \right) ^{2}}}\]
\[ L=\prod ^{n}_{i=1}f\left( y_{i}\right) =\left( 2\pi \right) ^{-\frac {n}{2}}\prod ^{n}_{i=1}i\sigma ^{-n}e^{-\frac {1}{2\sigma^2}\sum ^{n}_{i=1}\left( \frac {y_{i}-i\theta }{i}\right) ^{2}}\]
\[ \ln L=-\dfrac {n}{2}\ln 2\pi +\sum ^{n}_{i=1}\ln i-n\ln \sigma -\dfrac {1}{2\sigma ^{2}}\sum ^{n}_{i=1}\left( \theta -\dfrac {y_{i}}{i}\right) ^{2} \]
\[ \dfrac {\partial \ln L}{\partial \theta }=-\dfrac {1}{\sigma ^{2}}\sum ^{n}_{i=1}\left( \theta -\dfrac {y_{i}}{i}\right) =0\]
\[\therefore \hat {\theta }=\dfrac {1}{n}\sum ^{n}_{i=1}\dfrac {y_{i}}{i}\]
% \[ E\left( \hat {\theta }\right) =\dfrac {1}{n}\sum ^{n}_{i=1}\dfrac {E\left( y_{i}\right) }{i}=\dfrac {1}{n}.\sum ^{n}_{i=1}\dfrac {i\theta }{i}=\theta \]
\[ var\left( \hat {\theta }\right) =\dfrac {1}{n^{2}}\sum \dfrac {var\left( y_{i}\right) }{i^{2}}=\dfrac {1}{n^{2}}\sum ^{n}_{i=1}\dfrac {i^{2}\sigma ^{2}}{i^{2}}=\dfrac{\sigma ^{2}}{n} \]
\[ \because \dfrac {\partial ^{2}\ln L}{\partial \theta ^{2}}=-\dfrac {n}{\sigma ^{2}}\]
\[var\left( \hat {\theta }\right) \geq -\dfrac {1}{E\left( \dfrac {\partial ^{2}\ln L}{\partial \theta ^{2}}\right) }=\dfrac {\sigma ^{2}}{n}\]
\[ \therefore \hat{\theta}\,\, is\,\, an\,\, efficient\,\, estimator.\]


\begin{alignat*}{2}
    \sum_{i=1}^{4}\left( X_{i}-\bar {X}\right) ^{2}
    &=\sum ^{4}_{i=1}X^{2}_{i}-4\bar {X}^{2}\\
    &=\sum ^{4}_{i=1}X^{2}_{i}-4\Big[ \dfrac{1}{4}(X_1+X_2+X_3+X_4) \Big]^{2}\\  
    &=\sum ^{4}_{i=1}X^{2}_{i}-\dfrac {1}{2}\left( \sum ^{4}_{i=1}X^{2}_{i}+2\sum _{1\leq i < j\leq 4}X_{i}X_{j}\right) \\
    &=\dfrac {3}{4}\sum ^{4}_{i=1}X^{2}_{i}-\dfrac {1}{2}\sum _{1\leq i < j\leq 4}X_{i}X_{j}
\end{alignat*}
\begin{alignat*}{2}
    RHS  
    &=\Big( \dfrac {1}{2}X_1^{2}-X_{1}X_{2}+\dfrac {1}{2}X^{2}_{2} \Big)
     +\Big( \dfrac {2}{3}X^{2}_{3}+\dfrac {1}{6}\left( X_{1}+X_{2}\right) ^{2}-\dfrac {2}{3}X_{3}\left( X_{1}+X_{2}\right) \Big)
     +\dfrac{3}{4}\Big( X_4-\dfrac{1}{3}(X_1+X_2+X_3) \Big)^2\\
    &=\cdots \cdots(simple\, but\, tedious\, simplifications)\\
    &=\dfrac {3}{4}\sum ^{4}_{i=1}X^{2}_{i}-\dfrac {1}{2}\sum _{1\leq i < j\leq 4}X_{i}X_{j}
\end{alignat*}
\[\therefore 
\sum_{i=1}^{4}\left( X_{i}-\bar {X}\right) ^{2}=
\dfrac {\left( X_{1}-X_{2}\right) ^{2}}{2}+\dfrac {\left[ X_{3}-\frac {\left( X_{1}+X_{2}\right) }{2}\right] ^{2}}{\frac {3}{2}}
    \dfrac {\left[ X_{4}-\frac {\left( X_{1}+X_{2}+X_{3}\right) }{3}\right] ^{2}}{\frac {4}{3}}
  \]

\[\because X_1-X_2\sim N(0,\sqrt{2})\]
\[\therefore \dfrac{(X_1-X_2)^2}{2}\sim \chi_1^2\]
\[\because X_3-\dfrac{X_1+X_2}{2}\sim N(0,\sqrt{\dfrac{3}{2}})\]
\[\therefore \dfrac{(X_3-\frac{X_1+X_2}{2})^2}{\frac{3}{2}} \sim \chi_1^2\]
\[\because X_4-\dfrac{X_1+X_2+X_3}{3}\sim N(0,\sqrt{\dfrac{4}{3}})\]
\[\therefore \dfrac{(X_4-\frac{X_1+X_2+X_3}{3})^2}{\frac{4}{3}} \sim \chi_1^2\]

  \[ \boldsymbol{X}=
    \begin{pmatrix}X_1 & X_2 & X_3 &X_4 \end{pmatrix} '
    \]
\[
    \begin{pmatrix}X_1-X_2 \\ X_3-\frac{X_1+X_2}{2} \\ X_4-\frac{X_1+X_2+X3}{3} \end{pmatrix}
    = \begin{pmatrix} 1 & -1 & 0 & 0 \\
                     -\frac{1}{2} & -\frac{1}{2} & 1 & 0 \\
                     -\frac{1}{3} & -\frac{1}{3} & -\frac{1}{3} & 1
    \end{pmatrix} \begin{pmatrix}X_1 \\ X_2 \\ X_3 \\ X_4\end{pmatrix}
    =\boldsymbol{AX}
    \]
\[
    var( \boldsymbol{AX} ) = \boldsymbol{A}var(\boldsymbol{X})\boldsymbol{A}'
= \begin{pmatrix} 1 & -1 & 0 & 0 \\
                     -\frac{1}{2} & -\frac{1}{2} & 1 & 0 \\
                     -\frac{1}{3} & -\frac{1}{3} & -\frac{1}{3} & 1
    \end{pmatrix}
    \begin{pmatrix} 1 & 0 & 0 & 0 \\
                    0 & 1 & 0 & 0 \\
                    0 & 0 & 1 & 0 \\
                    0 & 0 & 0 & 1 \\
    \end{pmatrix}
    \begin{pmatrix} 1 & -\frac{1}{2} & -\frac{1}{3} \\
                    -1 & -\frac{1}{2} & -\frac{1}{3}\\
                    0 & 1 & -\frac{1}{3}\\
                    0 & 0 & 1\\
    \end{pmatrix}
=    \begin{pmatrix} 2 & 0 & 0 \\
                    0 & \frac{3}{2} & 0  \\
                    0 & 0 & \frac{4}{3}  \\
    \end{pmatrix}
\]
So the three terms in the RHS are independent each with a $\chi_1^2$ distribution.
\section*{Question g}
\[E\left( T-\theta \right) ^{2}=var\left( T\right) +B^{2}\]
\begin{alignat*}{2}
    var(T) &= var(\alpha_1\bar{X}+\alpha_2cS)\\
        &= \alpha_1^2var(\bar{X})+\alpha_2^2var(cS)\\
        &= \alpha_1^2\dfrac{\theta^2}{n}+\alpha_2^2(c^2-1)\theta^2
\end{alignat*}
\[B=E(T)-\theta=(\alpha_1+\alpha_2-1)\theta\]
\[\therefore E\left( T-\theta \right) ^{2} = \left[ \dfrac {\alpha ^{2}_{1}}{n}+\left( c^{2}-1\right) \alpha ^{2}_{2}+\left( \alpha _{1}+\alpha _{2}-1\right) ^{2}\right] \theta ^{2} \]
\begin{equation*}
\begin{cases}
    \dfrac {\partial E\left( T-\theta \right) ^{2}}{\partial \alpha _{1}}=\dfrac {2\alpha _{1}}{n}+2\left( \alpha _{1}+\alpha _{2}-1\right) =0 \\
    \dfrac {\partial E\left( T-\theta \right) ^{2}}{\partial \alpha _{2}}=2\left( c^{2}-1\right) \alpha _{2}+2\left( \alpha _{1}+\alpha _{2}-1\right) =0
\end{cases}
\end{equation*}
\begin{equation*}
    \therefore \begin{cases}
        \alpha_1 = \dfrac{n(c^2-1)}{(n+1)(c^2-1)+1}\\
        \alpha_2 = \dfrac{1}{(n+1)(c^2-1)+1}
    \end{cases}
\end{equation*}
\[A=\dfrac {\partial ^{2}E}{\partial \alpha_1^2}=\dfrac {2}{n}+2,B=\dfrac {\partial ^{2}E}{\partial \alpha _{1}\partial \alpha _{2}}=2,C=\dfrac {\partial ^{2}E}{\partial \alpha ^{2}_{2}}=2c^{2}\]
\[\because A<0,\,B^2-AC=4\left[1-\dfrac{n+1}{n}c^2\right]<0 \,(Using\,\, WolframAlpha) \]
\[\therefore E\left( T-\theta \right) ^{2} gets\,\, the\,\, minimum \,\,when\,\, \alpha_1\,\, and \,\,\alpha_2 \,\,equal\,\, the\,\, above\,\, values.\]
\[\therefore T=\dfrac{n(c^2-1)}{(n+1)(c^2-1)+1}\bar{X} + \dfrac{1}{(n+1)(c^2-1)+1}cS\,\, is\,\, the\,\, estimator\,\, that\,\, minimizes\,\, E\left( T-\theta \right) ^{2}\]

\end{document}