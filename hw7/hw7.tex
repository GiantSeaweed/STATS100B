%!TEX program = xelatex
\documentclass[a4papers]{ctexart}
%数学符号
\usepackage{amssymb}
\usepackage{amsmath}
%表格
\usepackage{graphicx,floatrow}
\usepackage{array}
\usepackage{booktabs}
\usepackage{makecell}
%页边距
\usepackage{geometry}
\geometry{left=2cm,right=2cm,top=2cm,bottom=2cm}

%首行缩进两字符 利用\indent \noindent进行控制
\usepackage{indentfirst}
\setlength{\parindent}{2em}

\setromanfont{Songti SC}
%\setromanfont{Heiti SC}

\title{STATS100B--Introduction to Mathematical Statistics \\Homework 7}
\author{Feng Shiwei \ UID:305256428}
\date{}
\begin{document}
\maketitle
\section*{Question a}
\noindent Solution:\\
\indent 
\[L=\dfrac{1}{(2\pi)^{\frac{n}{2}}}|\boldsymbol{\Sigma}|^{-\frac{1}{2}} 
    e^{\frac{1}{2}(\boldsymbol{Y}-\boldsymbol{\mu1})'\boldsymbol{\Sigma}^{-1}(\boldsymbol{Y}-\boldsymbol{\mu1}) }
\]
\[lnL = -\dfrac{n}{2}ln(2\pi\sigma^2)-\dfrac{1}{2}ln|V|-\dfrac{1}2\sigma^2(Y-\mu)V^{-1}(Y-\mu)
 \]
 \[\dfrac {\partial \ln L}{\partial \mu }=-\dfrac {1}{2\sigma ^{2}}\left[ -Y'V^{-1}1-1'V^{-1}Y+2\mu 1'V^{-1}1\right] =0
     \]
\[ \therefore \hat{\mu} = \dfrac{1'V^{-1}Y}{1'V^{-1}1}\]
\[ \dfrac {\partial \ln L}{\partial \sigma ^{2}}=-\dfrac {n}{2\sigma ^{2}}+\dfrac {1}{\sigma ^{4}}\left( Y-\mu 1\right) 'V^{-1}\left( Y-\mu 1\right) =0\]
\[\therefore \hat{\sigma^2} =\dfrac{\left( Y-\hat{\mu} 1\right) 'V^{-1}\left( Y-\hat{\mu} 1\right)}{n} \]


\section*{Question b}
\noindent  Solution:
\[E(\hat{\mu}) = E\left(\dfrac{1'V^{-1}Y}{1'V^{-1}1}\right) = \dfrac{1'V^{-1}E(Y)}{1'V^{-1}1} =\dfrac{1'V^{-1}\mu1}{1'V^{-1}1} =\mu\]
\[  \]



\section*{Question c}
\noindent Solution:
\[ \boldsymbol{I}(\boldsymbol{\theta}) = -E \begin{pmatrix}
    \dfrac {\partial ^{2}\ln L}{\partial \mu^2} 
    & \dfrac {\partial^{2}\ln L}{\partial \mu \partial \sigma ^{2}} \\ \\
    \dfrac {\partial ^{2}\ln L}{\partial \sigma ^{2}\partial \mu } 
    & \dfrac {\partial \ln L}{\partial {\sigma ^{2}}^{(2)} }
\end{pmatrix}\]
\[  \dfrac {\partial ^{2}\ln L}{\partial \mu^2 }=-\dfrac {1'V^{-1}1}{\sigma ^{2}} \]
\[ \dfrac {\partial^{2}\ln L}{\partial \mu \partial \sigma ^{2}}=\dfrac {1}{\sigma ^{4}}\left[ \mu 1'V^{-1}1-1'V^{-1}Y\right] \]
\[ \dfrac {\partial ^{2}\ln L}{\partial \sigma ^{2}\partial \mu } = \dfrac {1}{\sigma ^{4}}\left[ \mu 1'V^{-1}1-1'V^{-1}Y\right] \]
\[ \dfrac {\partial \ln L}{\partial {\sigma ^{2}}^{(2)} } =\dfrac {n}{2\sigma ^{4}}-\dfrac {1}{\sigma ^{6}}\left( Y-\mu 1\right) 'V^{-1}\left( Y-\mu 1\right)  \]
\[ \therefore \boldsymbol{I}(\boldsymbol{\theta}) =\begin{pmatrix}
    -\dfrac {1'V^{-1}1}{\sigma ^{2}}
\end{pmatrix}  \]

\section*{Question d}
\noindent Solution:
\begin{alignat*}{2}
    \sum_{i=1}^{4}\left( X_{i}-\bar {X}\right) ^{2}
    &=\sum ^{4}_{i=1}X^{2}_{i}-4\bar {X}^{2}\\
    &=\sum ^{4}_{i=1}X^{2}_{i}-4\Big[ \dfrac{1}{4}(X_1+X_2+X_3+X_4) \Big]^{2}\\  
    &=\sum ^{4}_{i=1}X^{2}_{i}-\dfrac {1}{2}\left( \sum ^{4}_{i=1}X^{2}_{i}+2\sum _{1\leq i < j\leq 4}X_{i}X_{j}\right) \\
    &=\dfrac {3}{4}\sum ^{4}_{i=1}X^{2}_{i}-\dfrac {1}{2}\sum _{1\leq i < j\leq 4}X_{i}X_{j}
\end{alignat*}
\begin{alignat*}{2}
    RHS  
    &=\Big( \dfrac {1}{2}X_1^{2}-X_{1}X_{2}+\dfrac {1}{2}X^{2}_{2} \Big)
     +\Big( \dfrac {2}{3}X^{2}_{3}+\dfrac {1}{6}\left( X_{1}+X_{2}\right) ^{2}-\dfrac {2}{3}X_{3}\left( X_{1}+X_{2}\right) \Big)
     +\dfrac{3}{4}\Big( X_4-\dfrac{1}{3}(X_1+X_2+X_3) \Big)^2\\
    &=\cdots \cdots(simple\, but\, tedious\, simplifications)\\
    &=\dfrac {3}{4}\sum ^{4}_{i=1}X^{2}_{i}-\dfrac {1}{2}\sum _{1\leq i < j\leq 4}X_{i}X_{j}
\end{alignat*}
\[\therefore 
\sum_{i=1}^{4}\left( X_{i}-\bar {X}\right) ^{2}=
\dfrac {\left( X_{1}-X_{2}\right) ^{2}}{2}+\dfrac {\left[ X_{3}-\frac {\left( X_{1}+X_{2}\right) }{2}\right] ^{2}}{\frac {3}{2}}
    \dfrac {\left[ X_{4}-\frac {\left( X_{1}+X_{2}+X_{3}\right) }{3}\right] ^{2}}{\frac {4}{3}}
  \]

\[\because X_1-X_2\sim N(0,\sqrt{2})\]
\[\therefore \dfrac{(X_1-X_2)^2}{2}\sim \chi_1^2\]
\[\because X_3-\dfrac{X_1+X_2}{2}\sim N(0,\sqrt{\dfrac{3}{2}})\]
\[\therefore \dfrac{(X_3-\frac{X_1+X_2}{2})^2}{\frac{3}{2}} \sim \chi_1^2\]
\[\because X_4-\dfrac{X_1+X_2+X_3}{3}\sim N(0,\sqrt{\dfrac{4}{3}})\]
\[\therefore \dfrac{(X_4-\frac{X_1+X_2+X_3}{3})^2}{\frac{4}{3}} \sim \chi_1^2\]

  \[ \boldsymbol{X}=
    \begin{pmatrix}X_1 & X_2 & X_3 &X_4 \end{pmatrix} '
    \]
\[
    \begin{pmatrix}X_1-X_2 \\ X_3-\frac{X_1+X_2}{2} \\ X_4-\frac{X_1+X_2+X3}{3} \end{pmatrix}
    = \begin{pmatrix} 1 & -1 & 0 & 0 \\
                     -\frac{1}{2} & -\frac{1}{2} & 1 & 0 \\
                     -\frac{1}{3} & -\frac{1}{3} & -\frac{1}{3} & 1
    \end{pmatrix} \begin{pmatrix}X_1 \\ X_2 \\ X_3 \\ X_4\end{pmatrix}
    =\boldsymbol{AX}
    \]
\[
    var( \boldsymbol{AX} ) = \boldsymbol{A}var(\boldsymbol{X})\boldsymbol{A}'
= \begin{pmatrix} 1 & -1 & 0 & 0 \\
                     -\frac{1}{2} & -\frac{1}{2} & 1 & 0 \\
                     -\frac{1}{3} & -\frac{1}{3} & -\frac{1}{3} & 1
    \end{pmatrix}
    \begin{pmatrix} 1 & 0 & 0 & 0 \\
                    0 & 1 & 0 & 0 \\
                    0 & 0 & 1 & 0 \\
                    0 & 0 & 0 & 1 \\
    \end{pmatrix}
    \begin{pmatrix} 1 & -\frac{1}{2} & -\frac{1}{3} \\
                    -1 & -\frac{1}{2} & -\frac{1}{3}\\
                    0 & 1 & -\frac{1}{3}\\
                    0 & 0 & 1\\
    \end{pmatrix}
=    \begin{pmatrix} 2 & 0 & 0 \\
                    0 & \frac{3}{2} & 0  \\
                    0 & 0 & \frac{4}{3}  \\
    \end{pmatrix}
\]
So the three terms in the RHS are independent each with a $\chi_1^2$ distribution.

\end{document}