%!TEX program = xelatex
\documentclass[a4papers]{ctexart}
%数学符号
\usepackage{amssymb}
\usepackage{amsmath}
%表格
\usepackage{graphicx,floatrow}
\usepackage{array}
\usepackage{booktabs}
\usepackage{makecell}
%页边距
\usepackage{geometry}
\geometry{left=2cm,right=2cm,top=2cm,bottom=2cm}

%首行缩进两字符 利用\indent \noindent进行控制
\usepackage{indentfirst}
\setlength{\parindent}{2em}

\setromanfont{Songti SC}
%\setromanfont{Heiti SC}

\title{STATS100B--Introduction to Mathematical Statistics \\Homework 1}
\author{Feng Shiwei \ UID:305256428}
\date{}
\begin{document}
\maketitle
\section*{Exercise 1}
\noindent Solution:\\
\indent 
$X\sim \Gamma(\alpha,\beta),$ so
\begin{alignat*}{2}
    f(x|(\alpha,\beta))&=\dfrac{x^{\alpha-1}e^{-\frac{x}{\beta}}}{\beta^\alpha\Gamma(\alpha)}\\
    &= \dfrac{1}{\beta^\alpha\Gamma(\alpha)}e^{(\alpha-1)lnx}e^{-\frac{x}{\beta}}\\
    &= \dfrac{1}{\beta^\alpha\Gamma(\alpha)}exp\{(\alpha-1)lnx-\frac{1}{\beta}x\}
\end{alignat*}
where $h(x)=1,\ c(\boldsymbol{\theta})=\dfrac{1}{\beta^\alpha\Gamma(\alpha)},\ 
\sum_{i=1}^{k}\left(w_i(\boldsymbol{\theta})t_i(x) \right)=(\alpha-1)lnx-\frac{1}{\beta}x.$

\section*{Exercise 2}
\noindent Solution:\\
\indent Let $F_Y(y)$ be the cdf of Y and suppose that $g$ is a monotonic reversible function.
\begin{alignat*}{2}
   F_Y(y) &= P(Y \le y)\\
          &= P(g(X) \le y)\\
          &= P(X \le g^{-1}(y))\\
          &= \int_{0}^{g^{-1}(y)}\dfrac{2}{\sqrt{2\pi}}e^{-\frac{1}{2}[g^{-1}(y)]^2}\,dx
\end{alignat*}
\begin{alignat*}{2}
    p_Y(y) &= F_Y(y)'
           &= \dfrac{2}{\sqrt{2\pi}}\frac{\mathrm{d}g^{-1}(y)}{\mathrm{d}y}e^{-\frac{1}{2}[g^{-1}(y)]^2}
\end{alignat*}
Because the pdf of gamma distribution is \[ p_Y(y) = \dfrac{y^{\alpha-1}e^{-\frac{y}{\beta}}}{\beta^\alpha\Gamma(\alpha)}, \]\,
let $g^{-1}(y)=c\sqrt{y}$,  where $c$ is constant.\\
Therefore, \[ -\dfrac{y}{\beta} = -\frac{1}{2}[g^{-1}(y)]^2 = -\frac{1}{2}c^2y \]
\[ \beta = \dfrac{2}{c^2} \]
\begin{alignat*}{2}
     p_Y(y)&=\dfrac{2}{\sqrt{2\pi}}\dfrac{1}{2\sqrt{y}}e^{-\frac{1}{2}[c\sqrt{y}]^2}\\
           &=\dfrac{y^{-\frac{1}{2}} e^{-\frac{c^2y}{2}}}{\sqrt{2}\sqrt{\pi}}\\
           &=\dfrac{y^{-\frac{1}{2}} e^{-\frac{y}{\beta}}}{\sqrt{2}\sqrt{\pi}}
\end{alignat*}
Let $\alpha = \dfrac{1}{2},\beta = 2$ which means $c = 1$ at the same time.\\
\begin{alignat*}{2}
     p_Y(y)&=\dfrac{y^{-\frac{1}{2}} e^{-\frac{y}{\beta}}}{\sqrt{2}\sqrt{\pi}}\\
           &=\dfrac{y^{-\frac{1}{2}} e^{-\frac{y}{2}}}{2^{\frac{1}{2}}\Gamma({\frac{1}{2}})}
\end{alignat*}
$\therefore$ When $Y=g(X)=\dfrac{1}{c^2}x^2=x^2,\, Y \sim \Gamma(\frac{1}{2},2).\\$

Actually, there are infinte transformations $Y=g(X)$ that can make Y a gamma distribution. 
When $Y=Cx^2$ where $C$ is a constant, $Y \sim \Gamma(\dfrac{1}{2},2C)$.

On the condition that $\alpha=\dfrac{1}{2},\beta=2,Y=x^2$,
\[
    E(X)=E(Y^{\frac{1}{2}})
    =\dfrac{ \Gamma(\alpha+\frac{1}{2}) \beta^{\frac{1}{2}} }{\Gamma(\alpha)} 
    =\dfrac{ \Gamma(1) 2^{\frac{1}{2}} }{\Gamma(\frac{1}{2})}=\sqrt{\dfrac{2}{\pi}}    
\]
\[
    E(X^2)=E(Y)
    =\dfrac{ \Gamma(\alpha+1) \beta^1 }{\Gamma(\alpha)} 
    =\dfrac{ \Gamma(\frac{3}{2}) 2^{1} }{\Gamma(\frac{1}{2})}
    =1
\]
\[
    var(X)=E(X^2)-(EX)^2=1-\dfrac{2}{\pi}
\]
\end{document}