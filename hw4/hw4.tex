%!TEX program = xelatex
\documentclass[a4papers]{ctexart}
%数学符号
\usepackage{amssymb}
\usepackage{amsmath}
%表格
\usepackage{graphicx,floatrow}
\usepackage{array}
\usepackage{booktabs}
\usepackage{makecell}
%页边距
\usepackage{geometry}
\geometry{left=2cm,right=2cm,top=2cm,bottom=2cm}

%首行缩进两字符 利用\indent \noindent进行控制
\usepackage{indentfirst}
\setlength{\parindent}{2em}

\setromanfont{Songti SC}
%\setromanfont{Heiti SC}

\title{STATS100B--Introduction to Mathematical Statistics \\Homework 4}
\author{Feng Shiwei \ UID:305256428}
\date{}
\begin{document}
\maketitle
\section*{Question a}
\noindent Solution:\\
\indent 
\[M_{\bar{X}}\left( t\right) =\left( 1-\dfrac {\beta }{n}t\right) ^{-n\alpha }\]
\begin{alignat*}{2}
    M_{\bar {X}}\left( \dfrac {2n}{\beta }t\right) &=\left( 1-\dfrac {\beta }{n}\times \dfrac {2n}{\beta }t\right) ^{-n\alpha } \\
    &= \left(1-2t\right)^{-\frac{2n\alpha}{2}}\\
    &= M_{\frac{2n}{\beta}\bar{X}}(t)
\end{alignat*}
So the transformation $\dfrac{2n}{\beta}\bar{X}$ follows $\chi^2$ distribution.\, The degree of freedom is $2n\alpha$. 

\section*{Question b}
\noindent  Solution:\\
$X\sim U(0,1),\,E(X)=\dfrac{1}{2},\,var(X)=\dfrac{1}{12}$.\\
\[E\begin{pmatrix}X \\ X^2 \end{pmatrix} = \begin{pmatrix}E(X) \\ E(X^2) \end{pmatrix}
    = \begin{pmatrix}\frac{1}{12} \\ E(X^2)+var(X) \end{pmatrix}
    = \begin{pmatrix}\frac{1}{2} \\ \frac{1}{3} \end{pmatrix}\]

\[ var\begin{pmatrix}X \\ X^2 \end{pmatrix} 
    =\begin{pmatrix} var(X) & cov(X,X^2) \\ cov(X,X^2) & var(X^2) \end{pmatrix}\]        
\begin{alignat*}{2}
    var(X^2) &= E[(X^2)^2]-[E(X^2)]^2 \\
    &= \int_0^1 x^4 dx -\left( \dfrac{1}{3} \right)^2\\
    &= \dfrac{4}{45}
\end{alignat*}
\begin{alignat*}{2}
    cov(X,X^2) &= E[X(X^2)]-E(X)E(X^2)\\
    &=\int_0^1 x^3 dx -\dfrac{1}{2}\times\dfrac{1}{3}\\
    &=\dfrac{1}{12}
\end{alignat*}
\[
    \therefore   var\begin{pmatrix}X \\ X^2 \end{pmatrix} = \begin{pmatrix}\frac{1}{12} & \frac{1}{12} \\ \frac{1}{12} & \frac{4}{45}  \end{pmatrix}
\]

\section*{Question c}
\noindent Solution:
\begin{alignat*}{2}
   E(Y) &= E(2\sqrt{X_1 X_2})\\
        &= 2E(\sqrt{X_1})E(\sqrt{X_2})
\end{alignat*}
\[\because E\left( X^{k}\right) =\dfrac {\Gamma \left( \alpha +k\right) \beta ^{k}}{\Gamma \left( \alpha \right) },\, X\sim \Gamma(\alpha,\beta)\]
\[\therefore E\left(\sqrt{{X_1}}\right) =\dfrac {\Gamma \left( \alpha +\dfrac{1}{2}\right)\cdot 1 ^{k}}{\Gamma \left( \alpha \right) } = \sqrt{\pi}\]
\[ E\left(\sqrt{{X_2}}\right) =\dfrac {\Gamma \left( \alpha + 1 \right) \cdot 1 ^{k}}{\Gamma \left( \alpha + \dfrac{1}{2}\right) } 
= \dfrac{\alpha! }{(\alpha-1)! \sqrt{\pi}}
= \dfrac{\alpha}{\sqrt{\pi}}\]
\[\therefore E(Y) = 2\times\sqrt{\pi}\times\dfrac{\alpha}{\pi}=2\alpha\]
\begin{alignat*}{2}
    var(Y)&=var(2\sqrt{X_1 X_2})\\
          &=E(Y^2)-\left(E(Y)\right)^2\\
          &=E(4X_1 X_2)-\left(E(Y)\right)^2\\
          &=4E(X_1)E(X_2)-\left(E(Y)\right)^2\\
          &=4\times\Big(\alpha\times 1\Big)\times\Big((\alpha+\dfrac{1}{2})\times 1\Big)-(2\alpha)^2\\
          &=2\alpha
\end{alignat*}



\section*{Question d}
\noindent Solution:
\[\begin{pmatrix}\bar{X}\\\bar{Y}\end{pmatrix} \sim 
    N_2\Bigg(\begin{pmatrix}\mu_1\\\mu_2\end{pmatrix},
        \begin{pmatrix}\sigma_1^2 & \rho \sigma_{1}\sigma_{2} \\ \rho \sigma_{1}\sigma_{2} & \sigma_2^2\end{pmatrix} \Bigg),\,
  \boldsymbol{\Sigma} = \begin{pmatrix}\sigma_1^2 & \rho \sigma_{1}\sigma_{2} \\ \rho \sigma_{1}\sigma_{2} & \sigma_2^2\end{pmatrix}
\]
\begin{alignat*}{2}
    M_{(\bar{X},\bar{Y})}(t_1,t_2)&=E(e^{t_1\bar{X}+t_2 \bar{Y}})\\
    &=E\Big(e^{t_1 \cdot \frac{1}{n}\sum_{i=1}^{n}X_i+t_2 \cdot \frac{1}{n}\sum_{j=1}^{n}Y_j } \Big)\\
    &=\prod_{i=1}^{n}E\Big(e^{\frac{t_1}{n}X_i+\frac{t_2}{n}Y_i}\Big)\\
    &=\Big(e^{\frac{\boldsymbol{t}'}{n}\mu+\frac{1}{2}\frac{\boldsymbol{t}'}{n} {\boldsymbol{\Sigma}} \frac{\boldsymbol{t}}{n}}\Big)^n\\
    &=e^{\boldsymbol{t}'\mu+ \frac{1}{2}\boldsymbol{t}'  \frac{\boldsymbol{\Sigma}}{n} \boldsymbol{t} }
\end{alignat*}
\[ \therefore \begin{pmatrix}\bar{X}\\\bar{Y}\end{pmatrix} \sim  
    N_2\Bigg( \begin{pmatrix}\mu_1\\\mu_2\end{pmatrix},\dfrac{\boldsymbol{\Sigma}}{n}  \Bigg)\]

\[ \therefore n(\bar{X}-\mu_1,\bar{Y}-\mu_2)\boldsymbol{\Sigma}^{-1}\begin{pmatrix}\bar{X}-\mu_1 \\ \bar{Y}-\mu_2 \end{pmatrix} 
 = (\bar{X}-\mu_1,\bar{Y}-\mu_2) \Big( \dfrac{\boldsymbol{\Sigma}}{n}\Big)^{-1}  \begin{pmatrix}\bar{X}-\mu_1 \\ \bar{Y}-\mu_2 \end{pmatrix} 
 \sim \chi_2^2
    \]



\end{document}